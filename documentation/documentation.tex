\documentclass[a4paper,10pt]{report}


% Title Page
\title{NLP project documentation}
\author{Alex Toney}


\begin{document}
\maketitle

\section{File Overview}
\begin{tabular}{ | l | l | }
\hline
debug.py	& For the pretty print debug function\\ \hline
grammar\_fsm.py  & Contains the FSM for the semantics\\ \hline
help.py         & Will contain help in the future\\ \hline
lg\_fsm.py       & The old finite state machine for link grammar\\ \hline
lg\_py.c         & The C module\\ \hline
lg\_test.py      & The core file currently\\ \hline
semantic\_rules.py & Contains all the semantic rules\\ \hline
semantics.py    & Parser to retrieve the semantic rules from RelEx\\ \hline
setup.py        & To compile the C module\\ \hline
\end{tabular}
\section{General Design}
\subsection{Semantic Matching}

For the semantic matching it has sets of rules, the input tags are run through 
a non-deterministic finite state machine that keeps left and right registers.
The rule sets are as follows, each rule has a regular expression to match to,
it also has a set of registers to match to and then set if they match.

\subsection{Semantic rule tokenizing}
I have another non-deterministic finite state machine(read: regex finite state
machine) to parse the rules that were taken from RelEx into something that is
manageable.

\section{Reference}
\subsection{Link Grammar Reference}
\begin{tabular}{ | l | l | }
	\hline
    A  & Attributive\\ \hline
    AA & AA is used in the construction "How big a dog was it?"\\ \hline
    AF & Connects adjectives to verbs in cases where the adjectiveis "fronted"\\ \hline
    B  & Is used in a number of situations, involving relative clauses and questions.\\ \hline
    D  & Connects determiners to nouns,\\ \hline
    EA & Connects adverbs to adjectives,\\ \hline
    EB & Connects adverbs to forms of "be" before an object, adjective, or prepositional phrase,\\ \hline
    I  & Connects certain verbs with infinitives,\\ \hline
    J  & Connects prepositions to their objects,\\ \hline
    M  & Connects nouns to various kinds of post-nominal modifiers without commas,\\ \hline
    Mv & Connects verbs (and adjectives) to modifying phrases,\\ \hline
    O* & Connects transitive verbs to direct or indirect objects,\\ \hline
    OX & Is a special object connector used for "filler" subjects like "it" and "there",\\ \hline
    Pp & Connects forms of "have" with past participles,\\ \hline
    Pa & Connects certain verbs to predicative adjectives,\\ \hline
    R  & Connects nouns to relative clauses,\\ \hline
    S  & Connects subject-nouns to finite verbs,\\ \hline
    Ss & Noun-verb Agreement,\\ \hline
    Sp & Noun-verb Agreement,\\ \hline
    Wd & Declarative Sentences,\\ \hline
    Wq & Questions,\\ \hline
    Ws & Questions,\\ \hline
    Wj & Questions,\\ \hline
    Wi & Imperatives,\\ \hline
    Xi & Abbreviations,\\ \hline
    Xp & Periods,\\ \hline
    Xx & Colons and semi-colons,\\ \hline
    Z  & Connects the preposition "as" to certain verbs,\\ \hline
\end{tabular}
\section{Other Projects}
\begin{enumerate}
 \item FreeLing -- http://www.lsi.upc.edu/~nlp/freeling/
 \item NLTK
 \item RelEx
 \item OpenCog
 \item Stanford Parser
\end{enumerate}

\end{document}          
